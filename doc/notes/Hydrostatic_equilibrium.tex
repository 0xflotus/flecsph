\documentclass[notes.tex]{subfiles}
 
\begin{document}

\section{Hydrostatic Equilibrium}
\label{sec:equi}

The initial data are based on a cubic lattice within a sphere of radius $R$.
The density function, based on radius, $\rho(\vec{r})$ is known using the result of the Lane-Emden equation ( we use polytropic index $n = 1$ here).
The mass associate to each particle $i$ of the total $N$ particles:

$$ m_i = \frac{\rho(\vec{r_i})}{n_r} \mbox{ with } n_r = \frac{3N}{4 \pi R^3}$$

The smoothing length is define constant and the same for all particles for all the simulation:

$$ h = \frac{1}{2}\sqrt{\frac{3N_N}{4\pi n}} $$

Here we choose $N_N$, the average number of neighbors, to be 100.

\subsection{Roche lobe problem}

To create Hydrostatic Equilibrium Models we use a different equation of motion. This version use Roche Lobe:

\begin{equation}
\frac{d\vec{v_i}}{dt} = \frac{\vec{F}_i^{Grav}}{m_i} + \frac{\vec{F}_i^{Hydro}}{m_i} + \vec{F}_i^{Roche} - \frac{\vec{v_i}}{t_{relax}}
\end{equation}


With $t_{relax} \leq t_{osc} \sim (G\rho)^{-1/2}$ and
where $\vec{F}_i^{Roche}$ is:

$$\vec{F}_i^{Roche} = \mu (3+q) x_i \hat{\vec{x}} + \mu q y_i \hat{\vec{y}}-\mu z_i \hat{\vec{z}}$$

With $\mu$ to be determined (for us $\mu = 0.069$) and $q = \frac{M'}{M}=1$ as the two polytropes have the same total mass.
This is apply to each star to get the equilibrium and the simulate the tidal effect.

\subsection{Darwin problem}

This is the way we use to generate the final simulation.
The equation of motion for the relaxation is now:
\begin{equation}
\label{eq:darwin}
\frac{d\vec{v_i}}{dt} = \frac{\vec{F}_i^{Grav}}{m_i} +\frac{\vec{F}_i^{Hydro}}{m_i} + \vec{F}_i^{Rot} - \frac{\vec{v_i}}{t_{relax}}
\end{equation}

With $t_{relax}$ same as before and $\vec{F}_{Rot}$ defined by:

\begin{equation}
\vec{F}_{Rot} = \Omega^2(x_i \vec{\hat{x}}+y_i\vec{\hat{y}})
\end{equation}

With $\Omega = \sqrt{\frac{G(M+M')}{a^3}}$.

Or $L_z = Q_{zz}\Omega$ and $Q_{zz} = \sum_i(x_i^2+y_i^2)$. At $t=0$ we compute the total angular moment $L_z$ which stay constant.
Using it during the relaxation we can compute $\Omega$ as: $\Omega = \frac{L_z}{Q_{zz}}$ just by recomputing $Q_{zz}$.

Here the scheme is in $N^2$ but just for the relaxation step.

For this relaxation we use two stars generated as before, applying equation of motion \ref{eq:darwin}.
Using $a$ as the distance between the two polytropes  (Here $a=2.9$ for $R=1$) and $\vec{\hat{x}}$ going for the center of the first to the second star, and $\vec{\hat{z}}$ is like the rotation vector.

%\section{Further-works}
%Here are just several open questions for this simulator:
%
%\begin{itemize}
%\item Is it possible to use tabulated EOS in this model ?
%\item How to implement adaptive smoothing length ?
%\item Is it interesting to start thinking about a grid over an octree for GR ?
%\item Is there another method than FMM for gravitation ? For efficient computation.
%\item Is adaptive $\Delta t$ forced ?
%\end{itemize}

% The smoothing length with $N_N$ the desire number of neighbors.
% The total volume in the sphere is $V = \frac{4}{3} \pi R^3$ with N particle in this volume so $ N / V$ particles per $cm^3$.
% If we need $N_N$ neighbor particles we need to find the volume $V_N$ containing the $N_N$ particles as $V_N = \frac{4}{3} \pi r^3$ as:
% $$ V_N \times \frac{N}{\frac{4\pi R^3}{3}} = N_N \Rightarrow \frac{4 \pi r^3}{3}  \times N \times \frac{3}{4\pi R^3} = N_N \Rightarrow r^3 = \frac{N_N R^3}{N} \Rightarrow r = \sqrt[3]{\frac{N_NR^3}{N}}  $$
% And h is half this radius:
% $$ h = \frac{1}{2}\sqrt[3]{\frac{N_NR^3}{N}} $$

%\section{Octree construction}
%
%With $n$ particles we need an array $a$ of size $n*2-1$.
%The algorithm is:
%
%\begin{algorithm}
%\caption{Construct tree}\label{construct}
%\begin{algorithmic}[1]
%\Procedure{Construct}{$particles$}
%\State $n \leftarrow $ Number of particles
%\State $tree \leftarrow $ Array of nodes size $2*n-1$
%\State $Init array $
%\While{$r\not=0$}\Comment{We have the answer if r is 0}
%\State $a\gets b$
%\State $b\gets r$
%\State $r\gets a\bmod b$
%\EndWhile\label{euclidendwhile}
%\State \textbf{return} $b$\Comment{The gcd is b}
%\EndProcedure
%\end{algorithmic}
%\end{algorithm}


\end{document}