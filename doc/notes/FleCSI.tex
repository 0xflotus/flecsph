\documentclass[notes.tex]{subfiles}
 
\begin{document}

\section{FleCSI}
FleCSI is a compile-time configurable framework designed to support
multi-physics application development.
As such, FleCSI provides a very general set of infrastructure design patterns
that can be specialized and extended to suit the needs of a broad variety of
solver and data requirements.
FleCSI currently supports multi-dimensional mesh topology, geometry, and
adjacency information, as well as n-dimensional hashed-tree data structures,
graph partitioning interfaces, and dependency closures.

FleCSI introduces a functional programming model with control, execution, and
data abstractions that are consistent both with MPI and with state-of-the-art,
task-based runtimes such as Legion, HPX and Charm++.
The abstraction layer insulates developers from the underlying runtime, while
allowing support for multiple runtime systems including conventional models
like asynchronous MPI.

The intent is to provide developers with a concrete set of user-friendly
programming tools that can be used now, while allowing flexibility in choosing
runtime implementations and optimization that can be applied to future
architectures and runtimes.

FleCSI's control and execution models provide formal nomenclature for
describing poorly understood concepts such as kernels and tasks.
FleCSI's data model provides a low-buy-in approach that makes it an attractive
option for many application projects, as developers are not locked into
particular layouts or data structure representations.

FleCSI currently provides a parallel but not distributed implementation of
binary, quad- and octree topologies.
Domain decomposition is implemented using space-filling curves, such as Morton
ordering curve.

At current stage, FleCSI framework requires implementation of a \textit{driver} 
and a \textit{specialization\_driver}.
The role of the \textit{specialization\_driver} is to provide the data model
and its parallel distribution.
Currently, FleCSI does not has this feature fully implemented, so we provide it.
The next step will be to incorporate it directly from FleCSPH to FleCSI as we
reach a good level of performance.
The \textit{driver} represents the general execution of the resolution without
worrying of the data locality and communications.
As FleCSI is a code in development, its structure may change in the future and
we keep track of these changes in FleCSPH.

\end{document}