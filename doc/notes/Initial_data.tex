\documentclass[notes.tex]{subfiles}
 
\begin{document}

\section{Initial Data}
\label{sec:equi}

Initial particle configurations are constructed using various methods,
including regular cubic lattices, random particle distributions, or sequences
of spherical shells. 
For a star, the density as a function of radius $\rho(\vec{r})$ can be found
by solving Lane-Emden equation, where we use polytropic index $n = 1$
(see Section~\ref{sec:Lane-Emden}).
We pick particles with variable mass, such that the mass of a particle is
computed as:
\begin{align}
  m_i = \frac{\rho(\vec{r_i})}{n_r} \mbox{ with } n_r = \frac{3N}{4 \pi R^3}
\end{align}  

The smoothing length is set to a constant and uniform for all particles:
\begin{align}
  h = \frac{1}{2}\sqrt{\frac{3N_N}{4\pi n}}
\end{align}

Here we choose $N_N$, the average number of neighbors, to be 100.

\section{Solving Lane-Emden Equation}
\label{sec:Lane-Emden}

Lane-Emden equation describes equilibrium configuration of a star. This is an
equation which determines the star density as a function of radius.

As we consider the star as a polytropic fluid, we use the equation of Lane-Emden which is a form of the Poisson equation:

\begin{equation}\label{eq_LaneEmden}
  \frac{d^2\theta}{d \xi^2}+ \frac{2}{\xi}\frac{d\theta}{d\xi}+\theta^n = 0
\end{equation}

With $\xi$ and $\theta$ two dimensionless variables.
There is only exact solutions for a polytropic index $n = 0.5$, $1$ and $2$.
In our work we use a polytropic index of $1$ which can correspond to a NS simulation.

For $n=1$ the solution of equation \ref{eq_LaneEmden} is:

\begin{equation}
\theta(\xi)=\frac{sin(\xi)}{\xi}
\end{equation}

We note $\xi_1 = \pi$, the first value of $\xi$ as $\theta(\xi) = 0$.
$\theta(\xi)$ is also defined as:
\begin{equation}
 \theta(\xi) = \Big(\frac{\rho(\xi)}{\rho_c}\Big)^{\frac{1}{n}}  = \frac{\rho(\xi)}{\rho_c}
\end{equation}

With $\rho_c$ the internal density of the star and $\rho$ the density at a determined radius. $\xi$ is defined as:
$$ \xi = Ar = \sqrt{\frac{4\pi G}{K(n+1)}\rho_c^{(n-1)/n}} \times r = \sqrt{\frac{2\pi G}{K}}\times r \mbox{ (for } n=1 \mbox{)}$$

With $K$ a proportionality constant.

From the previous equations we can write the stellar radius $R$ as:
\begin{equation}
R = \sqrt{\frac{K(n+1)}{4\pi G}}\rho_c^{(1-n)/2}\xi_1 = \sqrt{ \frac{K}{2\pi G} } \times \xi_1
\end{equation}

(We note that for $n=1$ the radius does not depend of the central density.)

If, for example, we use dimensionless units as $G=R=M=1$ (for the other results we use CGS with $G = 6.674 \times 10^{-8} cm^3g^{-1}s^{-2}$)
We can compute K as:
\begin{equation}
\label{eq:constant}
K = \frac{R^2  2 \pi G}{\xi_1^2}
\end{equation}

\begin{center}

\begin{tabular}{c|c|c|c|c|}
 & $NS_1$ & $NS_2$ & $NS_3$ & $NS_4$ \\
\hline
Radius (cm) & $R=G=M=1$ & 1500000 & 1400000 & 960000 \\
\hline
K & 0.636619 & 95598.00 & 83576.48 & 39156.94\\
\hline
\end{tabular}

\end{center}

Then we deduce the density function of $r$ as :

$$\rho(\xi) = \frac{sin(A\times r)}{A \times r} \times \rho_c \mbox{ with } A = \sqrt{\frac{2\pi G}{K}}
$$

As we know the total Mass $M$, the radius $R$ and the gravitational constant $G$ we can compute the central density as:

$$ \rho_c = \frac{M A^3}{4 \pi (sin(AR)-ARcos(AR)) } $$

Then we normalize the results to fit $R = M = G = 1$: $K' = K/(R^2G) $, $m_i' = m_i/M $, $h_i' = h_i / R$, $\vec{x_i}' = \vec{x_i}/R$


\subsection{Roche lobe problem}

To create Hydrostatic Equilibrium Models we use a different equation of motion. This version use Roche Lobe:

\begin{equation}
\frac{d\vec{v_i}}{dt} = \frac{\vec{F}_i^{Grav}}{m_i} + \frac{\vec{F}_i^{Hydro}}{m_i} + \vec{F}_i^{Roche} - \frac{\vec{v_i}}{t_{relax}}
\end{equation}


With $t_{relax} \leq t_{osc} \sim (G\rho)^{-1/2}$ and
where $\vec{F}_i^{Roche}$ is:

$$\vec{F}_i^{Roche} = \mu (3+q) x_i \hat{\vec{x}} + \mu q y_i \hat{\vec{y}}-\mu z_i \hat{\vec{z}}$$

With $\mu$ to be determined (for us $\mu = 0.069$) and $q = \frac{M'}{M}=1$ as the two polytropes have the same total mass.
This is apply to each star to get the equilibrium and the simulate the tidal effect.

\subsection{Darwin problem}

This is the way we use to generate the final simulation.
The equation of motion for the relaxation is now:
\begin{equation}
\label{eq:darwin}
\frac{d\vec{v_i}}{dt} = \frac{\vec{F}_i^{Grav}}{m_i} +\frac{\vec{F}_i^{Hydro}}{m_i} + \vec{F}_i^{Rot} - \frac{\vec{v_i}}{t_{relax}}
\end{equation}

With $t_{relax}$ same as before and $\vec{F}_{Rot}$ defined by:

\begin{equation}
\vec{F}_{Rot} = \Omega^2(x_i \vec{\hat{x}}+y_i\vec{\hat{y}})
\end{equation}

With $\Omega = \sqrt{\frac{G(M+M')}{a^3}}$.

Or $L_z = Q_{zz}\Omega$ and $Q_{zz} = \sum_i(x_i^2+y_i^2)$. At $t=0$ we compute the total angular moment $L_z$ which stay constant.
Using it during the relaxation we can compute $\Omega$ as: $\Omega = \frac{L_z}{Q_{zz}}$ just by recomputing $Q_{zz}$.

Here the scheme is in $N^2$ but just for the relaxation step.

For this relaxation we use two stars generated as before, applying equation of motion \ref{eq:darwin}.
Using $a$ as the distance between the two polytropes  (Here $a=2.9$ for $R=1$) and $\vec{\hat{x}}$ going for the center of the first to the second star, and $\vec{\hat{z}}$ is like the rotation vector.

%\section{Further-works}
%Here are just several open questions for this simulator:
%
%\begin{itemize}
%\item Is it possible to use tabulated EOS in this model ?
%\item How to implement adaptive smoothing length ?
%\item Is it interesting to start thinking about a grid over an octree for GR ?
%\item Is there another method than FMM for gravitation ? For efficient computation.
%\item Is adaptive $\Delta t$ forced ?
%\end{itemize}

% The smoothing length with $N_N$ the desire number of neighbors.
% The total volume in the sphere is $V = \frac{4}{3} \pi R^3$ with N particle in this volume so $ N / V$ particles per $cm^3$.
% If we need $N_N$ neighbor particles we need to find the volume $V_N$ containing the $N_N$ particles as $V_N = \frac{4}{3} \pi r^3$ as:
% $$ V_N \times \frac{N}{\frac{4\pi R^3}{3}} = N_N \Rightarrow \frac{4 \pi r^3}{3}  \times N \times \frac{3}{4\pi R^3} = N_N \Rightarrow r^3 = \frac{N_N R^3}{N} \Rightarrow r = \sqrt[3]{\frac{N_NR^3}{N}}  $$
% And h is half this radius:
% $$ h = \frac{1}{2}\sqrt[3]{\frac{N_NR^3}{N}} $$

%\section{Octree construction}
%
%With $n$ particles we need an array $a$ of size $n*2-1$.
%The algorithm is:
%
%\begin{algorithm}
%\caption{Construct tree}\label{construct}
%\begin{algorithmic}[1]
%\Procedure{Construct}{$particles$}
%\State $n \leftarrow $ Number of particles
%\State $tree \leftarrow $ Array of nodes size $2*n-1$
%\State $Init array $
%\While{$r\not=0$}\Comment{We have the answer if r is 0}
%\State $a\gets b$
%\State $b\gets r$
%\State $r\gets a\bmod b$
%\EndWhile\label{euclidendwhile}
%\State \textbf{return} $b$\Comment{The gcd is b}
%\EndProcedure
%\end{algorithmic}
%\end{algorithm}

\end{document}
