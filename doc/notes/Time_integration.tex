\documentclass[notes.tex]{subfiles}

\begin{document}

\section{Time Integration Scheme}

\begin{algorithm}
\caption{Leapfrog time integration}
\label{alg:leapfrog}
\begin{algorithmic}[1]
\State read initial state: {\tt \{r, v, u\}} $\leftarrow \{r^0, v^0, u^0\}$
\State compute {\tt rho(r), P(rho, u), cs(rho, P, u)} 
\State save $v^{1/2}$: {\tt v12 = v}
\State compute acceleration: {\tt a(P,rho,v12)} 
\State compute {\tt Dudt(P,rho,v,v12)} 
\While{iterations}
\State select timestep {\tt dt} $\leftarrow\Delta t$
\State {\tt v += a*dt/2}
\Comment velocity kick: $v^{n+1/2} = v^{n} + a^{n}\frac{\Delta t}2$

\State {\tt u += Dudt*dt/2}
\Comment internal energy kick:
         $u^{n+1/2} = u^{n} + \left(\frac{du}{dt}\right)^{n}\frac{\Delta t}2$

\State update neighbors
\State {\tt r += v*dt}
\Comment drift: $r^{n+1} = r^{n} + v^{n+1/2}\Delta t$

\State compute {\tt rho(r), P(rho, u), cs(rho, P, u)} 
\Comment uses $u^{n+1/2}$!

\State save $v^{n+1/2}$: {\tt v12 = v}
\State compute acceleration: {\tt a(P,rho,v12)} 
\Comment $a^{n+1} = a[P^{n+1},\rho^{n+1},\Pi_{ab}(v^{n+1/2})]$
\State {\tt v += a*dt/2}
\Comment second velocity kick: $v^{n+1} = v^{n+1/2} + a^{n+1}\frac{\Delta t}2$
\State update neighbors

\State compute {\tt Dudt(P,rho,v,v12)} 
\Comment $\left(\frac{du}{dt}\right)^{n+1} 
          = \dot{u}[P^{n+1},\rho^{n+1},v^{n+1},\Pi_{ab}(v^{n+1/2})]$
\State {\tt u += Dudt*dt/2}
\Comment second internal energy kick:
         $u^{n+1} = u^{n+1/2} + \left(\frac{du}{dt}\right)^{n+1}\frac{\Delta t}2$
\EndWhile
\end{algorithmic}
\end{algorithm}

For the time integrator, we are using the leapfrog algorithm, specifically its
"kick-drift-kick" variation (see Algorithm~\ref{alg:leapfrog}). This algorithm
belongs to the family of symplectic integrators and, in the absence of
gravity, conserves energy exactly.

Leapfrog time integrators are efficient for Hamiltonian systems, specifically
for particle simulations. The name comes from the fact that particle
velocities are updated at half-steps while the positions at integer steps, so
that the two leap over each other. A pair of updates from timestep $n$ to
$n+1$ has the following form:
\begin{align}
v^{n+1/2} &= v^{n-1/2} + a(r^{n}) \Delta t, \\
r^{n+1}   &= r^n + v^{n+1/2} \Delta t.
\end{align}
Note that in this simple form the accelerations $a(r^n)$ are computed
synchronously with positions and assumed independent from the velocities $v^n$.
This formulation is time-symmetric and reversible up to roundoff.

An equivalent "kick-drift-kick" formulation was shown to be stable for
variable time steps:
\begin{align}
v^{n+1/2} &= v^n + a(r^{n}) \Delta t/2,  \;\;\; &\text{"kick"} \\
r^{n+1}   &= r^n + v^{n+1/2} \Delta t,   \;\;\; &\text{"drift"}\\
v^{n+1}   &= v^n + a(r^{n+1}) \Delta t/2,\;\;\; &\text{"kick"}
\end{align}

In application to the basic SPH formulation 
(\ref{eq:basic-dudt}-\ref{eq:basic-dvdt}), this method reads:
\begin{align}
v^{n+1/2} = v^n &+ a[r^{n},\Pi_{ab}(v^{n-1/2})] \Delta t/2, 
\label{eq:kick_v1}\\
u^{n+1/2} = u^n 
          &+ \frac{du}{dt}\left[r^n,v^n,\Pi_{ab}(v^{n-1/2})\right]\Delta t/2,
\label{eq:kick_u1}\\
r^{n+1} = r^n &+ v^{n+1/2} \Delta t,\\
v^{n+1} = v^{n+1/2} &+ a\left[r^{n+1},\Pi_{ab}(v^{n+1/2})\right] \Delta t/2, 
\label{eq:kick_v2}\\
u^{n+1} = u^{n+1/2} 
       &+ \frac{du}{dt}\left[r^{n+1},v^{n+1},\Pi_{ab}(v^{n+1/2})
                       \right]\Delta t/2.
\label{eq:kick_u2}                       
\end{align}
where we highlighted variable dependencies between different time levels.

In this formulation, the internal energy is stored and updated synchronously
with the velocities rather than the coordinates.
This is done to achieve exact energy conservation:
\begin{align}
  E^{n+1/2} - E^{n-1/2} 
   &=
     \sum_a m_a \left[u_a^{n+1/2} - u_a^{n-1/2}\right]
   + \sum_a \frac12 m_a \left[\left(\vec{v}^{n+1/2}_a\right)^2 
                      - \left(\vec{v}^{n-1/2}_a\right)^2\right] 
   \\                   
   &= \sum_a m_a\left(\frac{du^{n}_a}{dt} 
    + \vec{v}^{n}_a\cdot\frac{d\vec{v}^n_a}{dt}\right),
   \;\;\;
   \text{where}\;
   \vec{v}_a^n := \frac{\vec{v}_a^{n-1/2} + \vec{v}_a^{n-1/2}}2.
   \label{eq:vn}
\end{align}

If we substitute corresponding time derivatives from
(\ref{eq:basic-dudt}-\ref{eq:basic-dvdt}), this expression vanishes
(as shown in e.g. Rosswog 2009, their Section 2.4):
\begin{align}
 E^{n+1/2} - E^{n-1/2} = 
 -\sum_{a,b} m_a m_b \left(
        \frac{P_a}{\rho_a^2} \vec{v}^{n}_b
      + \frac{P_b}{\rho_b^2} \vec{v}^{n}_a
      + \Pi_{ab}\frac{\vec{v}^{n}_a + \vec{v}^{n}_b}{2}
    \right)\cdot\nabla_a W^{n}_{ab} = 0.
\label{eq:energy-cons}    
\end{align}

For this expression to vanish, the viscosity tensors $\Pi_ab$ in
(\ref{eq:kick_v1}) and (\ref{eq:kick_u1}) above (or in (\ref{eq:kick_v2}) and
(\ref{eq:kick_v2})) should be identical. Even though they might depend on the
values of internal energy at previous half-step, $u^{n-1/2}$, it does not
affect tensor symmetry and energy conservation at half-steps.
Similarly, the pressure in (\ref{eq:energy-cons}) depends on the density at
the current integer step and internal energy at previous half-step, but it
does not violate energy conservation as long as the same value of the pressure
is used to compute accelerations and time derivatives of internal energy.
At the same time, when computing derivaties of internal energy using
expression (\ref{eq:basic-dudt}), the dot products 
$\vec{v}^{n}_{ab}\cdot\nabla_a W^n_{ab}$ must be computed with velocities
$\vec{v}^{n}_a$ at integer timesteps, as in (\ref{eq:vn}).

\subsection{Adaptive timestep}

The timestep can be adaptive and determined by
$\Delta t = \lambda_{\rm CFL}\min(\Delta t_1,\Delta t_2)$, where
$\lambda_{\rm CFL}\approx0.1$ is a Courant factor limit, and the timescales
$\Delta t_1$, $\Delta t_2$ are:
\begin{align}
  \Delta t_1 &= \min_a\left(\frac{h_a}{c_a(1 + 1.2\alpha)
                                      + 1.2\ \beta\ \max_b\mu_{ab}}\right), \\
  \Delta t_2 &= \min_a\sqrt{\frac{h_a}{|\dot\vec{v}_a|}},
\end{align}
where $c_a$ is a sound speed, $\alpha$ and $\beta$ are viscosity parameters,
and $\mu_{ab}$ is a viscosity function, as defined in equation (\ref{eq:visc_mu}).


\end{document}
