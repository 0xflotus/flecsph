\documentclass[notes.tex]{subfiles}
 
\begin{document}


\section{Time Integration Scheme}
\subsection{Leap-Frog Method}
Leap-frog time integration is frequently used in particle simulation. The name comes that the velocities are updated on half steps and the positions on integer steps so the two leap over each other. After computing accelerations, one step takes the form
\begin{align}
v^{i+1/2} = v^{i-1/2} + a^i \Delta t \\
r^{i+1} = r^i + v^{i+1/2} \Delta t
\end{align}
For $v^{i+1}$, a common approximation in SPH is to assume that the velocity at the current time plays a minor role in the computation of the acceleration (i.e. velocity changes are small with each time step) and then the following approximation can be made
\begin{equation}
v^{i+1} = \frac{1}{2} (v^{i-1/2}+v^{i+1/2})
\end{equation}
At the first step, we only have initial velocity so we need to follow below routing
\begin{align}
v^{1/2} = v^{0} + a^0 \Delta t/2 \\
r^{1} = r^0 + v^{1/2} \Delta t
\end{align}

The time step is adaptive and determined with $\Delta t = Min(\Delta t_1, \Delta t_2)$:

$$ \Delta t_1 = k \mbox{Min}_i(\frac{h_i}{c_i+1.2\alpha c_i + 1.2 \beta \mbox{Max}_j\mu_{ij}}) \mbox{ with } k \approx 0.1 $$
$$ \Delta t_2 = \mbox{Min}_i\sqrt{(\frac{h_i}{|\dot{\vec{v_i}}|})} $$


\end{document}
