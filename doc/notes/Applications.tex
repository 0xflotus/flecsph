\documentclass[notes.tex]{subfiles}
 
\begin{document}


\section{Applications}

\subsection{Sod Shock Tube}
The Sod shock tube is the test consists of a one-dimensional Riemann problem with the following initial parameters
\begin{equation}
(\rho, v, p)_{t=0} = \begin{cases}
(1.0,0.0,1.0) & \text{if} \indent 0 < x \leq 0.5 \\
(0.125,0.0,0.1) & \text{if} \indent 0.5 < x < 1.0
\end{cases}
\end{equation}
This link shows some references and values for Sod shock tube problem that also includes boundary and jump conditions(\url{http://www.phys.lsu.edu/~tohline/PHYS7412/sod.html}).

Also, we would like to re-generate the shock test result from Rosswog's paper. In that paper, he shows the result of a 2D relativistic shock tube test where the left state is given by $[P, v_x, v_y, N]_L = [40/3,0,0,10]$ and the right state by $[P, v_x, v_y, N]_R = [10^{-6},0,0,1]$ with $\Gamma = 5/3$

In our code, we use below parameters to get results

\subsection{Sedov Blast Wave}
A blast wave is the pressure and flow resulting from the deposition of a large amount of energy in a small very localized volume. This is another great test problem for computational fluid dynamics field.

There are different version of blast wave test but we consider the analytic solution for a point explosion is given by Sedov, making the assumption that the atmospheric pressure relative to the pressure insider the explosion negligible. The position of the shock as a function of time $t$, relative to the initiation of the explosion, is given by
\begin{equation}
R(t) = \left( \frac{e t^2}{\rho_0} \right)^{\frac{1}{\delta+2}}
\end{equation}
with $\delta = 2$ and $\delta = 3$ for cylindrical and spherical geometry respectively. The initial density $\rho_0$ whereas $e$ is a dimensionless energy. Right behind the shock we ahve the following properties
\begin{align}
\rho_2 = \frac{\Gamma +1}{\Gamma-1} \rho_0
P_2 = \frac{2}{\Gamma+1} \rho_0 w^2
v_2 = \frac{2}{\Gamma+1} w
\end{align}
where the shock velocity is
\begin{equation}
w(t) = \frac{d R}{dt} = \frac{2}{\delta+2} \frac{R(t)}{t}
\end{equation}

In numerical simulations, energy deposition in a single point is difficult to achieve. A solution to the problem is to make use of the bursting balloon analogue. Rather than depositing the total energy in a single point, the energy is released into a balloon of finite volume $V$
\begin{equation}
e = \frac{(P-P_0)V}{\Gamma -1}
\end{equation}
The energy release in a balloon of radius $r_0$ raises the pressure to the value
\begin{equation}
P = \frac{3(\Gamma-1)e}{(\delta+1) \pi r_0^{\delta}}
\end{equation}
Here, we test 2D blast wave test. In this simulation, we use ideal gas EOS with $\Gamma = 5/3$ and we are assuming that the undistributed area is at rest with a pressure $P_0 = 1.0 \time 10^{-5}$. The density is constant $\rho_0$, also in the pressurized region.

\end{document}
