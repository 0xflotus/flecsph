\documentclass{article}
\usepackage{amsmath, graphicx}
\usepackage{grffile}
\usepackage{dcolumn}
\usepackage{bm}
\usepackage{url}
\usepackage{epsfig}
\usepackage{mathrsfs}  %  package for the "curly" fonts 
%\usepackage{subfigure}
%\usepackage{multirow}
%\usepackage{epstopdf}
\usepackage{amsmath}
%\usepackage{algorithmicx}
\usepackage{amssymb}
%\usepackage{tensor}

\topmargin 0.30in
\textheight 9.00in

 \addtolength{\voffset}{-2cm}
 
\begin{document}

\title{Formulations and Equations for FleCSPH}

\author{Hyun Lim}

%\pacs{}
\maketitle

\section{Newtonian SPH Formulation}
By using the volume element $V_b = m_b / \rho_b$, we can formulate the Newtonian SPH scheme such that
\begin{align}
\rho_a = \sum_b m_b W_{ab} (h_a) \\
\frac{d u_a}{dt} = \frac{P_a}{\rho_a^2} \sum_b m_b \vec{v}_{ab} \cdot \nabla_a W_{ab} \\
\frac{d \vec{v}_a}{d t} = - \sum_b m_b \left(\frac{P_a}{\rho_a^2} + \frac{P_b}{\rho_b^2} \right) \nabla_a W_{ab}
\end{align}
where $W_{ab} = W(| \vec{r}_a - \vec{r}_b |,h)$. Further, we add artifical viscosity (or artificial dissipation) terms in SPH formulation such that
\begin{align}
\left(\frac{d u_a}{dt} \right)_{art} = \frac{1}{2} \sum_b m_b \Pi_{ab} \vec{v}_{ab} \cdot \nabla_a W_{ab}\\
\left(\frac{d\vec{v}_a}{dt} \right)_{art} = - \sum_b m_b \Pi_{ab}\nabla_a W_{ab}\\
\end{align}
In general, we can express the equations for internal energy and acceleration with artificial viscosity
\begin{align}
\frac{d u_a}{dt} = \sum_b m_b \left(\frac{P_a}{\rho_a^2} + \frac{\Pi_{ab}}{2} \right) \vec{v}_{ab} \cdot \nabla_a W_{ab} \\
\frac{d \vec{v}_a}{d t} = - \sum_b m_b \left(\frac{P_a}{\rho_a^2} + \frac{P_b}{\rho_b^2} + \Pi_{ab} \right) \nabla_a W_{ab}
\end{align}
$\Pi_{ab}$ may define different way but here we use
\begin{equation}
\Pi_{ab} = \begin{cases}
\frac{- \alpha \bar{c}_{ab} \mu_{ab} + \beta \mu_{ab}^2}{\bar{\rho}_{ab}} & \text{for $\vec{r}_{ab} \cdot \vec{v}_{ab} < 0$} \\
0 & \text{otherwise}
\end{cases}
\indent \text{where} \indent \mu_{ab} = \frac{\bar{h}_{ab} \vec{r}_{ab} \cdot \vec{v}_{ab}}{r^2_{ab} + \epsilon \bar{h}_{ab}^2}
\end{equation}
For speed of sound, $c_s$ the usual form is
\begin{equation}
c_s = \sqrt{\frac{\partial p}{\partial \rho}}
\end{equation}
For example, from the Newton-Laplace equation, $c = \sqrt{\frac{K_s}{\rho}}$ where $K_s$ is a coefficient of stiffness, the isentropic bulk modulus
The values of $\epsilon$, $\alpha$, and $\beta$ can be chosen differently. Here, we use $\epsilon = 0.01h^2$, $\alpha = 1.0$, and $\beta = 2.0$.
Now, we need to test some cases for performance of code
\section{Kernel}
There are many kernels for SPH. We use simple cubic spline kernel for our case. Other higher order kernels will be added soon. The Monaghan's cubic spline kernel is
\begin{equation}
W(\vec{r},h) = \frac{\sigma}{h^D} \begin{cases}
1-\frac{3}{2} q^2 + \frac{3}{4} & \text{if} \indent 0 \leq q \leq 1 \\
\frac{1}{4} (1-q)^3  & \text{if} \indent 1 \leq q \leq 2 \\
0 & \text{otherwise}
\end{cases}
\end{equation}
where $q = r/h$, $D$ is the number of dimensions and $\sigma$ is a normalization constant with the values
\begin{equation}
\sigma =  \begin{cases}
\frac{2}{3} & \text{for 1D}  \\
\frac{10}{7 \pi} & \text{for 2D} \\
\frac{1}{\pi} & \text{for 3D}
\end{cases}
\end{equation}
\section{Sod Shock Tube}
The Sod shock tube is the test consists of a one-dimensional Riemann problem with the following initial parameters
\begin{equation}
(\rho, v, p)_{t=0} = \begin{cases}
(1.0,0.0,1.0) & \text{if} \indent 0 < x \leq 0.5 \\
(0.125,0.0,0.1) & \text{if} \indent 0.5 < x < 1.0
\end{cases}
\end{equation}
This link shows some references and values for Sod shock tube problem that also includes boundary and jump conditions(\url{http://www.phys.lsu.edu/~tohline/PHYS7412/sod.html}).

Also, we would like to re-generate the shock test result from Rosswog's paper. In that paper, he shows the result of a 2D relativistic shock tube test where the left state is given by $[P, v_x, v_y, N]_L = [40/3,0,0,10]$ and the right state by $[P, v_x, v_y, N]_R = [10^{-6},0,0,1]$ with $\Gamma = 5/3$

In our code, we use below parameters to get results

\section{Sedov Blast Wave}
A blast wave is the pressure and flow resulting from the deposition of a large amount of energy in a small very localized volume. This is another great test problem for computataional fluid dynamics field. 

There are different version of blast wave test but we consider the analytic solution for a point explosion is given by Sedov, making the assumption that the atmospheric pressure relative to the pressure insider the explosion negligible. The position of the shock as a function of time $t$, relative to the initiation of the explosion, is given by
\begin{equation}
R(t) = \left( \frac{e t^2}{\rho_0} \right)^{\frac{1}{\delta+2}}
\end{equation}
with $\delta = 2$ and $\delta = 3$ for cylindrical and spherical geometry respectively. The initial density $\rho_0$ whereas $e$ is a dimensionless energy. Right behind the shock we ahve the following properties
\begin{align}
\rho_2 = \frac{\Gamma +1}{\Gamma-1} \rho_0
P_2 = \frac{2}{\Gamma+1} \rho_0 w^2
v_2 = \frac{2}{\Gamma+1} w
\end{align}
where the shock velocity is
\begin{equation}
w(t) = \frac{d R}{dt} = \frac{2}{\delta+2} \frac{R(t)}{t}
\end{equation}

In numerical simulations, energy deposition in a single point is difficult to achieve. A solution to the problem is to make use of the bursting balloon analogue. Rather than depositing the total energy in a single point, the energy is released into a balloon of finite volume $V$
\begin{equation}
e = \frac{(P-P_0)V}{\Gamma -1}
\end{equation}
The energy release in a balloon of radius $r_0$ raises the pressure to the value
\begin{equation}
P = \frac{3(\Gamma-1)e}{(\delta+1) \pi r_0^{\delta}}
\end{equation}
Here, we test 3D blast wave test. In this simulation, we use ideal gas EOS with $\Gamma = 5/3$ and we are assuming that the undistributed area is at rest with a pressure $P_0 = 1.0 \time 10^{-5}$. The density is constan $\rho_0$, also in the pressurized region.

We anticipate the blast wave expolsition outgoing in octant $ 0<x,y,z<5$

\section{Equations of State}
To understand the inner property of stars, one needs to find the equation which describes the relation between the pressure of matter and its density, temperautre and other compositions such that
\begin{equation}
P = P(\rho, T, Y_e, ...)
\end{equation}
First, we consider analytic equations of state that are relevant for binary neutron stars
\subsection{Analytic EOS}
\subsubsection{Ideal Gas}
Ideal gas equation of state is
\begin{equation}
P(\rho,u) = (\Gamma - 1) \rho u
\end{equation}
where $\Gamma$ is the adiabatic index of the gas. For a monatomic gas, we set $\Gamma = 5/3$. For another test such as sod tube, people use $\Gamma = 1.4$ 
\subsubsection{Piecewise Polytrope}
For more relevant simulation, we choose piecewise polytropic EOS.(Ideal gas EOS is still good for many simple test cases like Sod tube) In our case, we assume constant entropy so that many thermodynamic situations can be approximated as polytropes or piecewise functions made up of polytropes.

For neutron star case, we assume degnerated Fermi gas of neutrons then polytropic constant for a non-relativistic dengerated neutron gas is
\begin{equation}
K_0 = \frac{(3 \pi^2)^{2/3} \hbar^2}{5 m_n^{8/3}}
\end{equation}
where $m_n$ is the mass of a proton and $\hbar$ is a Planck constant. For polytropic index, we set
\begin{equation}
\gamma_0 = \frac{5}{3}
\end{equation}
In the relativistic case,
\begin{equation}
\gamma_1 = \frac{5}{2}
\end{equation}
Then, piecewise polytrope EOS is
\begin{equation}
P(\rho) = \begin{cases}
K_0 \rho^{\gamma_0} & \text{if} \indent \rho \leq \rho_0 \\
\frac{K_0 \rho_0^{\gamma_0}}{\rho_0^{\gamma_1}} \rho^{\gamma_1} & \text{if} \indent \rho > \rho_0 
\end{cases}
\end{equation}
where $\rho_0 = 5 \times 10^{14} g/cm^3$. We can combine the piecewise polytropic EOS with ideal gas to attain an EOS vaild at both low and high densities 
\subsubsection{Maxwell-Boltzmann}
\subsubsection{Helmholtz}
\subsection{Tabulated EOS}

\section{Time Integration Scheme}
\subsection{Leap-Frog Method}
Leap-frog time integration is frequently used in particle simulation. The nams comes that the velocities are updated on half steps and the positions on integer steps so the two leap over each other. After computing accelerations, one step takes the form
\begin{align}
v^{i+1/2} = v^{i-1/2} + a^i \Delta t \\
r^{i+1} = r^i + v^{i+1/2} \Delta t
\end{align}
For $v^{i+1}$, a common approximation in SPH is to assume that the velocity at the current time plays a minor role in the computation of the acceleration (i.e. velocity changes are small with each time step) and then the following approximation can be made
\begin{equation}
v^{i+1} = \frac{1}{2} (v^{i-1/2}+v^{i+1/2})
\end{equation}
At the first step, we only have initial velocity so we need to follow below routing
\begin{align}
v^{1/2} = v^{0} + a^0 \Delta t/2 \\
r^{1} = r^0 + v^{1/2} \Delta t
\end{align}
\section{Boundary Condition}

\end{document}
